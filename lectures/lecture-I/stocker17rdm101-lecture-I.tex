\documentclass{beamer}
\usetheme{Boadilla}
\usecolortheme{sidebartab}

\usepackage{hyperref}
\usepackage{showexpl} 
\usepackage{graphicx}
\usepackage{tikz}
\usepackage{color}
\usepackage{siunitx}
\usepackage[version=3]{mhchem}
\usepackage{chemfig}
\usepackage{changes}
\usepackage[many]{tcolorbox}
\usepackage{natbib}
\bibliographystyle{unsrtnat}
\setcitestyle{square,numbers}

\beamertemplatenavigationsymbolsempty
\setbeamertemplate{footline}{}
\setbeamertemplate{bibliography item}{\insertbiblabel}

\lstloadlanguages{[LaTeX]Tex} 
\lstset{% 
     basicstyle=\ttfamily\large, 
     commentstyle=\itshape\ttfamily, 
     showspaces=false, 
     showstringspaces=false, 
     breaklines=true, 
     breakautoindent=false, 
     captionpos=t,
     explpreset={numbers=none},
     pos=b
} 

\newtcolorbox{titlebox}[1][]{
	width=1.2\textwidth,
	boxrule=0pt,
	leftrule=0pt,
	rightrule=0pt,
	toprule=0pt,
	bottomrule=0pt,
	arc=0pt,
	colback=yellow!20,
	opacityback=0.7,
	enhanced jigsaw
}

\title{Research Data and Data Management Planning}
\author{Markus Stocker}
\date{September 12, 2017}

\begin{document}

\maketitle

\begin{frame}
  \frametitle{Outline}
  
  \begin{itemize}
  \item What are research data
  \item Research data lifecycle
  \item Data types, formats, models and standards
  \item Metadata
  \item Data management, plans and planning tools
  \end{itemize}
\end{frame}

% Joan Miro (1968). Landscape. Acrylic on canvas
% Essentially a white canvas with a colored dot
% It makes for a good example for what a datum is, at least for philosophers
{
	\usebackgroundtemplate{ %
		\begin{tikzpicture}[remember picture, overlay]%
		\node at (current page.center) {\includegraphics[height=\paperheight]{graphics/joan-miro-landscape-1968.jpg}};%
		\end{tikzpicture}%
	}%
	\setbeamertemplate{navigation symbols}{}
	\begin{frame}[plain]
	\end{frame}
}

% Datum is ultimately reducible to a lack of uniformity (between signals, symbols, ...)
% That's the reason why lights used to signal emergency flash
% Data depend on the occurrence of differences
% Defined as x being distinct from y
% Where x, y are uninterpreted (typed) variables
% Variable is a symbol that acts as a placeholder for an unknown (or changeable) referent
% Referent: the thing in the world that the symbol denotes (or stands) for
{
	\usebackgroundtemplate{ %
		\begin{tikzpicture}[remember picture, overlay]%
		\node at (current page.center) {\includegraphics[height=\paperheight]{graphics/joan-miro-landscape-1968.jpg}};%
		\end{tikzpicture}%
	}%
	\setbeamertemplate{navigation symbols}{}
	\begin{frame}[plain]
	\vspace{6cm}
			\begin{tikzpicture}
				\hspace{-0.55cm}
				\node {
					\begin{titlebox}
						\large Datum is ultimately reducible to a lack of uniformity \cite{floridi11philosophy}
					\end{titlebox}
				};
			\end{tikzpicture}
	\end{frame}
}

\begin{frame}
  \frametitle{Define data}
  
  \begin{itemize}
  \item Entities, physical or digital, used as evidence of phenomena \cite{borgman15big}
  \item A reinterpretable representation of information \cite{ccsds12oais}
  \item ...
  \item There is no consensus definition
  \item Even institutions that curate data may not define what they curate
  \end{itemize}
\end{frame}

\begin{frame}
  \frametitle{Data examples}
  
  \begin{itemize}
  \item Not just spreadsheets of numbers, also
  \item Sequences of bits
  \item Characters on a page
  \item Recording of sounds
  \item Physical and biological specimens
  \item Images
  \item Software
  \end{itemize}
\end{frame}

\begin{frame}
  \frametitle{Define research data}
  
  \begin{itemize}
  \item Unsurprisingly, there is no consensus on the definition
  \item Factual material [...] necessary to validate research findings \cite{epsrc17researchdata}
  \item Everything needed to reproduce a given scientific output \cite{surkis15researchdata}
  \item ...
  \end{itemize}
\end{frame}

\begin{frame}
  \frametitle{Research data examples}
  
  \begin{itemize}
  \item In addition to the obvious, e.g. data files
  \item Notebooks, e.g. laboratory, field, diaries, ...
  \item Questionnaires, audio and video tapes
  \item Models and scripts
  \item Workflows and protocols
  \end{itemize}
\end{frame}

\begin{frame}
  \frametitle{Research data lifecycle}
  
  \begin{itemize}
  \item 
  \end{itemize}
\end{frame}

% http://www2.le.ac.uk/services/research-data/documents/UoL_ReserchDataDefinitions_20120904.pdf
\begin{frame}
  \frametitle{Data types}
  
  \begin{itemize}
  \item 
  \end{itemize}
\end{frame}

\begin{frame}
  \frametitle{Data formats}
  
  \begin{itemize}
  \item 
  \end{itemize}
\end{frame}

\begin{frame}
  \frametitle{Data models}
  
  \begin{itemize}
  \item 
  \end{itemize}
\end{frame}

\begin{frame}
  \frametitle{Standards}
  
  \begin{itemize}
  \item 
  \end{itemize}
\end{frame}

\begin{frame}
  \frametitle{Metadata}
  
  \begin{itemize}
  \item 
  \end{itemize}
\end{frame}

\begin{frame}
  \frametitle{Data management}
  
  \begin{itemize}
  \item 
  \end{itemize}
\end{frame}

\begin{frame}
  \frametitle{Planning data management}
  
  \begin{itemize}
  \item 
  \end{itemize}
\end{frame}

\begin{frame}
  \frametitle{Tools for data management planning}
  
  \begin{itemize}
  \item 
  \end{itemize}
\end{frame}

\begin{frame}
  \frametitle{Take aways}
  
\end{frame}

\begin{frame}
  \frametitle{References}
  \tiny
  \bibliography{../../../../../bibliography/bibliography}
  
  Slide 3: Joan Mir{\'o} (1968). Landscape. Acrylic on canvas. Fundaci{\'o}n Joan Mir{\'o}, Barcelona. \url{https://www.fmirobcn.org/en/colection/catalog-works/5442/p-landscape-p}
\end{frame}

\end{document}
